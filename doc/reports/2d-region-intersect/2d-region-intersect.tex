\documentclass[twocolumn]{article}

\usepackage[xcolor]{rvdtx}

%\usepackage[fontsize=8pt,baseline=9.6pt,lines=50]{grid}
%\usepackage[fontsize=9pt,baseline=10.8pt]{grid}
\usepackage[fontsize=10pt,baseline=12pt,lines=53]{grid}
%\usepackage[fontsize=11pt,baseline=13.2pt]{grid}
%\usepackage[fontsize=12pt,baseline=14.4pt]{grid}
%\usepackage[fontsize=20pt,baseline=24pt,lines=20]{grid}

\newcommand{\ip}[2]{(#1, #2)}
\columnsep=20pt
\begin{document}

\title{ViperVM -- 2D Region Intersection}
\author{Sylvain HENRY}
\contact{hsyl20@gmail.com}
\version{1.1}
\date{2014/10/26}
%\keywords{\LaTeX, grid typesetting}


\maketitle

\section{Introduction}

In ViperVM, we need to know if two given memory regions overlap in order to
perform data transfers and kernel executions concurrently. The difficulty is to
detect overlapping for 2D regions (e.g. matrices with padding).

\subsection{Definitions}

In ViperVM, 1D and 2D regions are defined as:
\begin{itemize}
   \item Region1D(o,w)
   \item Region2D(o,w,h,p)
\end{itemize}
where
\begin{description}
   \item[o] stands for "offset"
   \item[w] stands for "width", i.e. the number of cells (number of cells in a line in the 2D case)
   \item[h] stands for "height", i.e. the number of lines
   \item[p] stands for "padding", the number of cells between two lines in the 2D
   case
\end{description}

\section{Intersections}

We want to know if two regions have at least one cell in common. The easy case
is when both regions are 1D regions. Otherwise, if one of the region is 1D, it
can be converted into a 2D region (with h = 1 and p = 0) and the algorithm to
detect the intersection of two 2D regions is used.

\subsection{1D regions}

Two 1D regions r1(o1,w1) and r2(o2,w2) do not overlap iff 
\[ o1 \ge o2+w2 \] or \[o2 \ge o1 + w1 \]

\subsection{2D regions}

Suppose we want to know if two 2D regions r1(o1,w1,h1,p1) and r2(o2,w2,h2,p2)
overlap.

First we can check if their covering 1D regions overlap or not. If they do not,
then the 2D regions neither do. The covering 1D region for $r_{2d}(o,w,h,p)$ is:

\[ \textrm{cover}(r_{2d}) = r_{1d}(o, h \times (w + p) - p) \]

Otherwise, we need to find if there is an overlap of two lines from the two
regions. The naive way to do it would imply $h1 \times h2$ 1D region overlap
tests (one for each couples of lines from r1 and r2) so we need to do something
more clever.

\subsubsection{Cycle}

The first thing to do is to find a \emph{cycle}, that is a range of lines in r1
and r2 that are repeated. If we find there is no overlapping on the cycle, there
won't be any in the whole regions. To find the cycle, we need to count the number
of lines of r1 and the number of lines of r2 before the relative offsets between
the two line beginnings is the same. Hence we use the least common multiple
(lcm) to find the width of a cycle:

\[
\begin{array}{rcl}
cw &=& \textrm{lcm} (w1+p1,w2+p2) \\
   &=& m1 \times (w1 + p1) \\ 
   &=& m2 \times (w2 + p2) \\
\end{array}
\]

Then, for each region we need to compute \textbf{ro}, the relative offset
(alignment) of the region in $\mathbb{Z}/\mathbb{Z}cw$.

\[ ro_i  = o_i\ \textrm{mod}\ (w_i + p_i) \]

Finally we can compute the lines of the cycles. $fs$ is the first truncated line
(if any) and $ls$ is the last truncated line (if any). $os$ are the other
full lines.

\noindent If $ro \leq p$ then

\[
\begin{array}{rcl}
fs &=& \varnothing\\
ls &=& \varnothing\\
os &=& \{(ro+k\times (w+p), w), k \in [0,m-1]\}
\end{array}
\]

\noindent else ($ro > p$)

\[
\begin{array}{rcl}
fs &=& \{(0, ro-p)\}\\
ls &=& \{(cw-(w+p)+ro, w+p-ro)\}\\
os &=& \{(ro+k\times (w+p), w), k \in [0,m-2]\}
\end{array}
\]

For a region, the lines in the cycle are:
\[ lines = fs \cup ls \cup os \]

If we test each couple of lines from two regions, we need at most $(m1+1) \times
(m2+1)$ tests. The problem is now that these tests may return false positives
because we don't check if the overlapping part of the cycle exists in practice.
Indeed the cycle can be larger than the real regions because we don't use the
numbers of lines nor the offsets to restrict it when we build it. We do it in
the next section.

\subsubsection{Filtering}

To remove false positives, we need to filter out parts of the cycle that don't
exist in the real case.

We need to consider the intersection of the covering 1D regions:.

\[
\begin{array}{rcl}
c1               &=& \textrm{max}\ o1\ o2 \\
c2               &=& (\textrm{min}\ (o1+w1)\ (o2+w2)) - 1\\
icw              &=& c2 - c1 + 1\\
r1 \cap r2       &=& r_{1d}(c1, icw)
\end{array}
\]

First, if the width of the intersection of the covering 1D regions $icw$
is larger or equal to the cycle width, then the whole cycle has to be
taken into account.

Otherwise, we compute the offsets of the covering region beginning (c1) and end
(c2) in $\mathbb{Z}/\mathbb{Z}cw$ (respectively rc1 and rc2).
\[
\begin{array}{rcl}
rc1 &=& c1 \textrm{ mod } cw \\
rc2 &=& c2 \textrm{ mod } cw
\end{array}
\]

There are two cases:
\begin{enumerate}
   \item $rc1 \le rc2$ it means that the covering region is included into the cycle:
   only the range [rc1,rc2] of the cycle is \textbf{valid}
   \item $rc1 > rc2$ it means that the covering region overlaps the cycle border:
   the range ]rc2,rc1[ of the cycle is \textbf{invalid}
\end{enumerate}

These predicates have to be used to filter out lines during the overlapping
tests for each couple of lines in the cycle.


\section{Conclusion}

We have provided an algorithm to detect 2D region intersection. While it may
seem computationally heavy, in practice we expect most regions to be of similar
shapes (power of two, etc.) hence leading to a reduced cost.

\end{document}  

% End of document.
